\documentclass{article}
\usepackage[utf8]{inputenc}

\title{Emotion challenge fact sheet}
\author{iCV}
\date{February 2017}

\begin{document}

\maketitle

\section{Team details}

\begin{itemize}
\item Team name

CBSR-CASIA

\item Team leader name

Jianzhu, Guo
\item Team leader address, phone number and email

Address: No.95 Zhongguancun East Road, Haidian District, Beijing 100190, China
Email: guojianzhu1994@foxmail.com
Phone: 86-15600118066
\item Rest of the team members

Shuai, Zhou \\
Jilin, Wu \\
Jun, Wan \\
Xiangyu, Zhu \\
Zhen, Lei 
\item Team website URL (if any)

http://www.cbsr.ia.ac.cn/users/jwan/research.html 
\item Affiliation

National Laboratory of Pattern Recognition (NLPR), Institute of Automation, Chinese Academy of Sciences (CASIA).
\end{itemize}

\section{Contribution details}

\begin{itemize}
\item Title of the contribution
\item Final score
\item General method description

We fisrt extract the each ID's lamdmark feature, and subsract it's mean, then we get each ID's landmark offset of $R^{136\times1}$. And we use an modified AlexNet to extract the $R^{256\times1}$ feature of the $224\times224$ image on the second-last full connected layer. Then we concact the two features, and connect to a full connected layer. We use Hinge Loss to train the net, the ground truth is a single label indicating the complementary and dominant emotion.
\item References
\item Representative image / diagram of the method
\item Describe data preprocessing techniques applied (if any)
\end{itemize}


\section{Face Landmarks Detection}
\subsection{Features / Data representation}
Describe features used or data representation model FOR FACE LANDMARKS DETECTION (if any)

We use Dlib open source library to extract the landmark.

\subsection{Dimensionality reduction}
Dimensionality reduction technique applied FOR FACE LANDMARKS DETECTION (if any)

\subsection{Compositional model}
Compositional model used, i.e. pictorial structure FOR FACE LANDMARKS DETECTION (if any)

\subsection{Learning strategy}
Learning strategy applied FOR FACE LANDMARKS DETECTION (if any)

\subsection{Other techniques}
Other technique/strategy used not included in previous items FOR FACE LANDMARKS DETECTION (if any)

\subsection{Method complexity}
Method complexity FOR FACE LANDMARKS DETECTION


\section{Dominant emotion recognition}
\subsection{Features / Data representation}
Describe features used or data representation model FOR DOMINANT EMOTION RECOGNITION (if any)

\subsection{Dimensionality reduction}
Dimensionality reduction technique applied FOR DOMINANT EMOTION RECOGNITION (if any)

\subsection{Compositional model}
Compositional model used, i.e. pictorial structure FOR DOMINANT EMOTION RECOGNITION (if any)

\subsection{Learning strategy}
Learning strategy applied FOR DOMINANT EMOTION RECOGNITION (if any)

\subsection{Other techniques}
Other technique/strategy used not included in previous items FOR DOMINANT EMOTION RECOGNITION (if any)

\subsection{Method complexity}
Method complexity FOR DOMINANT EMOTION RECOGNITION


\section{Complementary emotion recognition}
\subsection{Features / Data representation}
Describe features used or data representation model FOR COMPLEMENTARY EMOTION RECOGNITION (if any)

\subsection{Dimensionality reduction}
Dimensionality reduction technique applied FOR COMPLEMENTARY EMOTION RECOGNITION (if any)

\subsection{Compositional model}
Compositional model used, i.e. pictorial structure FOR COMPLEMENTARY EMOTION RECOGNITION (if any)

\subsection{Learning strategy}
Learning strategy applied FOR COMPLEMENTARY EMOTION RECOGNITION (if any)

\subsection{Other techniques}
Other technique/strategy used not included in previous items FOR COMPLEMENTARY EMOTION RECOGNITION (if any)

\subsection{Method complexity}
Method complexity FOR COMPLEMENTARY EMOTION RECOGNITION


\section{Joint dominant and complementary emotion recognition}
\subsection{Features / Data representation}
Describe features used or data representation model FOR JOINT DOMINANT AND COMPLEMENTARY EMOTION RECOGNITION (if any)

Landmark offset and AlextNet feature extraction.
\subsection{Dimensionality reduction}
Dimensionality reduction technique applied FOR JOINT DOMINANT AND COMPLEMENTARY EMOTION RECOGNITION (if any)

\subsection{Compositional model}
Compositional model used, i.e. pictorial structure FOR JOINT DOMINANT AND COMPLEMENTARY EMOTION RECOGNITION (if any)

\subsection{Learning strategy}
Learning strategy applied FOR JOINT DOMINANT AND COMPLEMENTARY EMOTION RECOGNITION (if any)

\subsection{Other techniques}
Other technique/strategy used not included in previous items FOR JOINT DOMINANT AND COMPLEMENTARY EMOTION RECOGNITION (if any)

\subsection{Method complexity}
Method complexity FOR JOINT DOMINANT AND COMPLEMENTARY EMOTION RECOGNITION


\section{Global Method Description}

\begin{itemize}
\item Total method complexity: all stages
\item Which pre-trained or external methods have been used (for any stage, if any)
\item Which additional data has been used in addition to the provided training and validation data (at any stage, if any)
\item Qualitative advantages of the proposed solution

If takes advantage of the geometry feature and texture pattern.
\item Results of the comparison to other approaches (if any)
\item Novelty degree of the solution and if is has been previously published
\end{itemize}

\section{Other details}

\begin{itemize}
\item Language and implementation details (including platform, memory, parallelization requirements)

We use Python with OpenCV and Dlib package and Caffe deep learning framework.
\item Detailed list of prerequisites for compilation
\item Human effort required for implementation, training and validation?
\item Training/testing expended time?

Training takes about an hour on Titan-X GPU, validation of 7000 images takes about half minute.
\item General comments and impressions of the challenge?

The task is hard, and it's hard to find effective texture feature to do this challenge.
\end{itemize}
\end{document}
